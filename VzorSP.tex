\documentclass[12pt]{report}			% Začátek dokumentu
\usepackage{SP}							% Import stylu

\author{Jonáš Havelka}
\title{Neuronová síť}
\date{19. února 2020}
\vedouci{Dr. rer. nat. Michal Kočer}
\place{V Českých Budějovicích}
\skolnirok{2019/2020}
\logo{\includegraphics[scale=0.75]{logo_gymji.jpg}}

\newcommand{\Kotlin}{\icon{Kotlin-logo.png}Kotlin}
\newcommand{\git}{\underlineicon{Git-logo.png}\,git}
\newcommand{\GitHub}{\bigunderlineicon{GitHub-logo.png}\,GitHub}
\newcommand{\Gradle}{\bigicon{Gradle-logo.png}Gradle}

\begin{document}

	\mytitlepage						% Vygenerování titulní strany
	
	\prohlaseni{
		Prohlašuji, že jsem tuto práci vypracoval samostatně s vyznačením všech použitých pramenů.
	}	
	
	\abstrakt{
		\lipsum[1]						% Abstrakt
	}{
		\lipsum[1]						% Klíčová slova
	}
	
	\podekovani{
		\lipsum[2]						% Poděkování
	}
	
	\tableofcontents\newpage			% Obsah
	
	
	
	
	\chapter*{Úvod}
	
		\lipsum[1]	
	
	
	\part{Teoretická část}
	
		\chapter{Programovací jazyky a nástroje}
		
			\section{\protect\Kotlin}

				\Kotlin\ je programovací jazyk, který vznikl v roce 2011 jako moderní verze Javy. Na jeho vývoji se podílí hlavně firma JetBrains, avšak celý projekt je pod licencí Apache Licence (verze 2.0)\footnote{viz \odkaz{https://www.apache.org/licenses/LICENSE-2.0}} a veřejně přístupný na platformě \GitHub\footnote{viz \odkaz{https://github.com/JetBrains/kotlin}}.
				
				
				
				\subsection{Syntaxe}
				
					
				
				\subsection{Koma}
				
					Koma je vědecká matematická knihovna napsaná v \Kotlin u, zaměřená na vývoj multiplatformních numerických aplikací cílených na \gls{JVM}, \gls{JS} nebo/a na nativní kód).%https://koma.kyonifer.com/
					
					Zatím (k \date) je zaměřená hlavně na lineární algebru, především na n-rozměrné tenzory (a tedy i na matice, jež jsou 2-rozměrné tenzory) a operace s nimi. Dále při použití v \gls{JVM} a \gls{JS} dovoluje i vykreslování grafů.
				
				
			\section{\protect\Gradle}
			
				
				
				\subsection{Základní prvky}
			
			\section{\protect\git}
			
				
				
				\subsection{Příkazy}
					
					Pro ovládání \git u využíváme příkazovou řádku, tedy každému příkazu předchází slovo \code{git}, čímž specifikujeme, že následující příkaz má vykonávat \git .
					
					Abychom vůbec mohli začít, potřebujeme založit repozitář. To uděláme příkazem \code{git init}. Jakmile máme vytvořený repozitář, můžeme do něj přidat soubory. Jeden soubor přidáme \code{git add "název souboru.přípona"}, pokud chceme přidat všechny soubory (až na soubory definované v .gitignore viz níže), použijeme parametr: \code{git add -a}. Stejným příkazem se poté přidávají i změny v souborech.
					
					Stále jsme ale nevytvořili žádnou \uv{verzi} souborů. K tomu slouží \code{git commit}, popřípadě \code{git commit -m "Zpráva ke commitu"}. To, které soubory jsme už přidali, které již byly v commitu, co se od posledního commitu změnilo atd, si můžeme zobrazit  za pomoci \code{git status}. Výpis vypadá přibližně následovně:
					\begin{lstlisting}[autogobble=true]
						On branch my_work
						Your branch is up to date with 'origin/my_work'.

						Changes not staged for commit:
						  (use "git add <file>..." to update what will be committed)
						  (use "git restore <file>..." to discard changes in working directory)
						        modified:   SP.sty
						        modified:   VzorSP.tex
						        modified:   Zkratky.tex
						
						Untracked files:
						  (use "git add <file>..." to include in what will be committed)
						        images/Git-logo.png
						        images/GitHub-Mark.zip
						        images/GitHub-logo.png
						        images/Gradle-logo.png
						        images/Kotlin-logo.png
						
						no changes added to commit (use "git add" and/or "git commit -a")
					\end{lstlisting}
					
					Na prvním řádku vidíme název větve, ve které se nacházíme, na druhém, zda se poslední verze shoduje s poslední verzí publikovaného repozitáře. 7. - 9. řádek pak udává změny na souborech, které již byly v commitu, 13. - 17. udává soubory, které ještě nebyly přidány za pomocí \code{add}. Můžeme si všimnout, že \git\ nám radí, co máme použít za příkazy.
				
				\subsection{\protect\GitHub}
	
	
		
		
		\chapter{Strojové učení a umělá inteligence}
		
		\chapter{Neuronové sítě}
	
			\section{Historie}
			
			\section{Laický náhled}
			
			\section{Matematika}
			
		
	\part{Praktická část}



	\appendix
	\addcontentsline{toc}{part}{Apendix}
	
	\chapter*{Závěr}
	
		\lipsum[1]
	
	\nocite{*}
    \printbibliography					% Vytvoří seznam literatury
	\addcontentsline{toc}{chapter}{Bibliografie}
    \printglossary[title={Zkratky}]		% Vytvoří seznam zkratek
    
    \begin{prilohy}
    	\pitem{Fotky z pokusů}
    	\eitem{Vlastní program}
    	\eitem{Dokumentace}
    	\eitem{Testovací data}
    \end{prilohy}
\end{document}