\documentclass[12pt]{report}			% Začátek dokumentu
\usepackage{SP}							% Import stylu

\author{Jonáš Havelka}
\title{Neuronová síť}
\date{19. února 2020}
\vedouci{Dr. rer. nat. Michal Kočer}
\place{V Českých Budějovicích}
\skolnirok{2019/2020}
\logo{\includegraphics[scale=0.75]{logo_gymji.jpg}}

\begin{document}

	\mytitlepage						% Vygenerování titulní strany
	
	\prohlaseni{
		Prohlašuji, že jsem tuto práci vypracoval samostatně s vyznačením všech použitých pramenů.
	}	
	
	\abstrakt{
		\lipsum[1]						% Abstrakt
	}{
		\lipsum[1]						% Klíčová slova
	}
	
	\podekovani{
		\lipsum[2]						% Poděkování
	}
	
	\tableofcontents\newpage			% Obsah
	
	
	
	
	\chapter*{Úvod}
	
		\lipsum[1]	
	
	
	\part{Teoretická část}
	
		\chapter{Asymptotická notace}
			
			\section{O-notace}
				Odkaz v závorkách: \parencite[see][page 900]{einstein}\\
				Odkaz: \cite{knuthwebsite}\\
				A odkaz pod čarou: \footcite[see][s. 42]{latexcompanion}\\
				Dobrý den, ahoj, \gls{atd}\\
				Praha, \gls{tj} hlavní město ČR
				
			\section{Abeceda Abeceda Abeceda Abeceda Abeceda Abeceda Abeceda Abeceda Abeceda Abeceda }
		
	\part{Praktická část}



	\appendix
	\addcontentsline{toc}{part}{Apendix}
	
	\chapter*{Závěr}
	
		\lipsum[1]
	
	\nocite{*}
    \printbibliography					% Vytvoří seznam literatury
	\addcontentsline{toc}{chapter}{Bibliografie}
    \printglossary[title={Zkratky}]		% Vytvoří seznam zkratek
    
    \begin{prilohy}
    	\pitem{Fotky z pokusů}
    	\eitem{Vlastní program}
    	\eitem{Dokumentace}
    	\eitem{Testovací data}
    \end{prilohy}
\end{document}