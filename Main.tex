\documentclass[12pt]{report}			% Začátek dokumentu
\usepackage{SP}							% Import stylu

\author{Jonáš Havelka}
\title{Neuronová síť}
\date{19. února 2020}
\vedouci{Dr. rer. nat. Michal Kočer}
\place{V Českých Budějovicích}
\skolnirok{2019/2020}
\logo{\includegraphics[scale=0.75]{logo_gymji.jpg}}

\newcommand{\Kotlin}{\icon{Kotlin-logo.png}\gls{Kotlin}}
\newcommand{\git}{\underlineicon{Git-logo.png}\,git}
\newcommand{\GitHub}{\bigunderlineicon{GitHub-logo.png}\,GitHub}
\newcommand{\Gradle}{\bigicon{Gradle-logo.png}Gradle}

\newcommand{\R}{\mathbb{R}}   			% množina reálnách čísel
\newcommand{\Z}{\mathbb{Z}}   			% množina celých čísel
\newcommand{\N}{\mathbb{N}}   			% množina přirozených čísel
\newcommand{\powerset}[1]{\mathcal{P} ( #1 )}   
										% potenční množina -- množina všech podmnožin

\begin{document}

	\mytitlepage						% Vygenerování titulní strany
	
	\prohlaseni{
		Prohlašuji, že jsem tuto práci vypracoval samostatně s vyznačením všech použitých pramenů.
	}	
	
	\abstrakt{
	
		Neuronové sítě se dnes objevují všude, ať už je to vyhledávání, překládání, nebo třeba jen zpracovávání dat. Mnoho programovacích jazyků má své knihovny pro práci s umělou inteligencí, ale zrovna \Kotlin{}, který lze použít skoro kdekoliv (Webové stránky, servery, mobily) takovou knihovnu postrádá. 
		% Abstrakt
	}{
		Neuronové sítě, \Kotlin, Umělá inteligence, Multiplatformní knihovna						% Klíčová slova
	}
	
	\podekovani{
		Poděkování patří hlavně mému učiteli informatiky, který je zároveň vedoucím mé práce, za skvělou výuku na hodinách a velkou trpělivost při kontrole našich prací. Také nesmím zapomenout na Alžbětu Neubauerovou, který mě celý rok podporovala a práci několikrát provedla korekturu mé práce.
		
		Dále bych rád poděkoval všem komunitám, jejichž nástroje jsem používal, tj. JetBrains, v jejichž programovacím jazyce \Kotlin{} programuji a jejichž prostředí IntelliJ k tomu využívám, \Gradle{}, které používám ke kompilaci, \LaTeX{}, ve kterém píšu text a dále \git{} a \GitHub{}, jež uchovávají má data, ať už text nebo knihovnu. 					% Poděkování
	}
	
	\tableofcontents\newpage			% Obsah
	
	
	
	
	\chapter*{Úvod}
	
		Neuronové sítě jsou v poslední době velmi skloňované téma. Nikdo vlastně pořádně neví, jak fungují, ale fungují\footnote{Tady se hodí podotknout, že alespoň malou představu máme, přece jenom matematicky je to pouze sestup po gradientu, ale překvapivě to dokáže velmi mnoho}. Cílem této práce však nebude zkoumat neuronové sítě, ale implementovat je v co největším rozsahu (ať už struktury bez širšího využití jako asociativní paměť nebo často používané konvoluční sítě na rozpoznávání obrázků).
		
		\Kotlin je ideální programovací jazyk pro vývoj knihovny, protože umožňuje tuto knihovnu používat jak pro \gls{JVM}
		
		Celá maturitní práce je k dispozici na \GitHub{}u, text včetně zdrojového \LaTeX{}u na adrese \url{https://github.com/JoHavel/Maturitni-Seminarni-Prace/tree/my\_work} a knihovna samotná pak na \url{https://github.com/JoHavel/NeuralNetwork}.
	
	
	\part{Teoretická část}
		
		\chapter{Strojové učení a umělá inteligence}
		
		\chapter{Neuronové sítě}
		
			\section{Laický náhled}
			
				\subsection{Neuron}
					Počítačové neuronové sítě nejsou jen výmysl lidí, jejich základ nalezneme nervových soustavách živočichů. Základní stavební jednotka takové soustavy (stejně tak i neuronové sítě) je neuron. Neuron funguje tak, že přes \gls{dendrit}y přijímá elektrické (přesněji iontové) signály od jiných neuronů a když součet signálů přeteče určitou danou mez, vyšle neuron signál přes \gls{axon}y dál do dalších neuronů.
					
					Přenos signálu z \gls{axon}u do \gls{dendrit}u se odehrává v malých prostorách mezi nimi zvaných \gls{synapse}. Vodivost synapsí je ovlivněna jejich chemickým složením, a proto se domníváme, proces učení probíhá měněním těchto chemických spojů \autocite[s. 491]{Book:Informatika}.
					
					Náš umělý neuron tedy bude mít seznam \gls{dendrit}ů (nesoucích informaci, z jakého neuronu vedou signál a jak ho mění \gls{synapse}), tzv. aktivační funkci (tedy jak silný signál posílá dále v závislosti na součtu vstupních signálů) a výstupní signál. Často navíc bude obsahovat základní hodnotu (angl. bias), která reprezentuje hladinu iontů v neuronu.
				
				\subsection{Sítě}
					Jelikož nahodilé neurony by se těžko udržovali v paměti a operace na nich by byly velmi pomalé, potřebujeme síť nějak uspořádat. Nejjednodušším uspořádáním jsou vrstvy. Každý neuron z nějaké vrstvy má \gls{dendrit}y ze všech neuronů z vrstvy minulé. Tak se předejde cyklům, které jsou složité na výpočty, a navíc si nemusíme u každého neuronu pamatovat, ze kterých neuronů do něj vede signál.
					
				\subsection{Dopředná propagace a zpětná propagace}
					Dopředná propagace (častěji se používá anglický výraz forward propagation) je jednoduše spočítání signálů ve všech neuronech. Tedy u každého neuronu se sečtou vstupní signály (popř. přičte bias) a spočítá se funkční hodnota aktivační funkce v tomto bodě.
					
					Naopak zpětná propagace (častěji se používá anglický výraz backward propagation) je na základě chyby, kterou spočítáme z výstupu neuronové sítě a předpokládaného výstupu, určit, které proměnné hodnoty (\gls{synapse} a biasy) se na ní nejvíce podílejí. Potom tyto hodnoty posuneme odpovídajícím způsobem (stejně jako příroda mění chemické vlastnosti \gls{synapse})
			
			
			\section{Formální náhled}
				Nechť $n_i = (\left\{w_{x,\,i}\right\},\,f_i,\,b_i,v_i)$ je neuron, kde $w_{x,\,i} \in \R$ je hodnota (angl. weight), kterou \gls{synapse} jdoucí z neuronu $n_x$ násobí signál, $f_i: \R \leftarrow \R$ je aktivační funkce, $b_i \in \R$ je bias a $v_i \in \R$ je signál vycházející z daného neuronu. Potom dopředná propagace (tedy spočítání $v_i$) vypadá takto:
				$$ v_i = f_i\left(b_i + \sum_x v_x \cdot w_{x,\,y} \right) $$
				Což lze zapsat vektorově jako:
				$$ \vec{v} = (v_{x_1},\,v_{x_2},\,\ldots) $$
				$$ \vec{w}_i = (w_{x_1,\,i},\,w_{x_2,\,i},\,\ldots) $$
				$$ v_i = f_i\left(b_i + \vec{w}_i \cdot \vec{v} \right) $$
				Nebo můžeme do hodnot \uv{zakomponovat} i bias:
				$$ \vec{v} = (v_{x_1},\,v_{x_2},\,\ldots,\,1) $$
				$$ \vec{w}_i = (w_{x_1,\,i},\,w_{x_2,\,i},\,\ldots,\,b_i) $$
				$$ v_i = f_i\left(\vec{w}_i \cdot \vec{v} \right) $$
				
			
		
	\part{Praktická část}
	
		\chapter{Struktura knihovny}
			Knihovna je rozdělena do dvou částí:
			\begin{itemize}
				\item První, a ta hlavní, je core (česky jádro), které obsahuje definice neuronových sítí (tj. konvoluční neuronovou síť, obyčejnou neuronovou síť, asociativní paměť) a definice pro ně potřebné (například aktivační funkce)
				\item Druhá je mnistDatabase, která se stará o učení neuronových sítí na datových databázích.
			\end{itemize}
			
			\section{core}
			
			\section{mnistDatabase}
				\subsection{Databáze MNIST}
					\uv{Databáze MNIST, databáze ručně psaných číslic dostupná na stránkách \url{http://yann.lecun.com/exdb/mnist/} obsahuje 60\,000 tréninkových a 10\,000 ověřovacích příkladů. MNIST je vytvořená z databáze spravované NIST (National Institute of Standarts and Technology). Číslice mají normalizovanou velikost a jsou vycentrované v obrázcích shodné velikosti.} \parencite[přeloženo]{online:MNIST}
					
					Tuto databázi jsem použil pro první testování BasicNeuralNetwork, jelikož má pro první testování dostačující velikost. Pro pozdější testování využívám převážně EMNIST.
				\subsection{Databáze EMNIST}
					\uv{Databáze MNIST se stala standardem pro učení umělého vidění. Databáze MNIST je odvozená z databáze NIST Special Database 19, která obsahuje ručně psané číslice a velká i malá písmena. EMNIST (Extended MNIST), varianta celé databáze NIST, přebírá uspořádání z databáze MNIST\footnote{Má však prohozené řádky a sloupce obrázků.}.} \parencite[přeloženo]{article:EMNIST}
					
					Tato databáze obsahuje více příkladů než MNIST, navíc obsahuje i sety s písmeny, proto jsem po prvních pokusech s MNIST přešel na tuto databázi.

		\chapter{Používání knihovny}


	\appendix
	\addcontentsline{toc}{part}{Apendix}
	
	\chapter*{Závěr}
	
		\lipsum[1]
	
	\nocite{*}
    \printbibliography					% Vytvoří seznam literatury
	\addcontentsline{toc}{chapter}{Bibliografie}
    \printglossary[title={Slovníček pojmů}]	% Vytvoří seznam zkratek
    
    \begin{prilohy}
    	\pitem{Fotky z pokusů}
    	\eitem{Vlastní program}
    	\eitem{Dokumentace}
    	\eitem{Testovací data}
    \end{prilohy}
\end{document}