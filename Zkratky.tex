\newglossaryentry{Kotlin}{name={Kotlin},description={programovací jazyk vyvíjený firmou JetBrains, založen na Javě}}
\newglossaryentry{JVM}{name={JVM},description={Java Virtual Machine je virtuální stroj, který umožňuje běh Java Bytecodu, kódu, do kterého se překládá Java a Kotlin}}

\newglossaryentry{interface}{name={interface},description={česky rozhraní je v objektově orientovaném programování zabalení funkcí a vlastností \gls{trid}y, které by měla každá \gls{trid}a z nějaké skupiny mít (např. každá fronta by měla mít funkci pro přidání a odebrání prvku a jedna z jejích vlastností je velikost)}}
\newglossaryentry{trid}{name={třída},description={anglicky class je základní prvek objektově orientovaného programování. Obsahuje funkce a vlastnosti, které bude mít objekt, který se vytvoří z dané třídy (popřípadě třídy, jež budou z této třídy dědit)}}


\newglossaryentry{synapse}{name={synapse},description={spojení (mezera) mezi \gls{axon}em a \gls{dendrit}em, jež podle svých chemických vlastností zesílí nebo zeslabí signál předávaný z \gls{axon}u do \gls{dendrit}u}}

\newglossaryentry{dendrit}{name={dendrit},description={výběžek vedoucí signál do neuronu}}
\newglossaryentry{axon}{name={axon},description={výběžek vedoucí signál z neuronu}}
